% Created 2025-05-28 Wed 14:45
\documentclass[compress,dvipdfmx,11pt]{beamer}
\usepackage[T1]{fontenc}
\usepackage{graphicx}
\usepackage{amsmath}
\usepackage[normalem]{ulem}
\usepackage{hyperref}
\title[2025年 大崎研究室ゼミ]{\bf はじめてのスライド作り}
\author[]{山近 駿}
\institute{関西学院大学 工学部 情報工学課程}
\date{2016 年 5 月 28 日}
\setlength{\parskip}{1.5ex}
\renewcommand{\textbf}{\alert}
\usetheme{Ohsaki}
\begin{document}

\maketitle

\newcommand{\pivec}{\mathbf \pi}
\newcommand{\xvec}{\mathbf x}
\newcommand{\yvec}{\mathbf y}
\newcommand{\zvec}{\mathbf z}
\newcommand{\Emat}{\mathbf E}
\newcommand{\Imat}{\mathbf I}

\bf
\section{感想}
\label{sec:org6dcaea9}
\begin{frame}[label={sec:org066d3fe}]{今日の授業の感想}
\begin{itemize}
\item pdfファイルを作る際はlatexでファイルを作ってそれを \alert{変換} することで作成
できるということが分かった。また、スライドもpdfの一種なので同じよう
にして作成ができるということが分かった。
\end{itemize}
\end{frame}
\end{document}
